%-----------------------------------------------------------------------------%
\chapter{\kesimpulan} % Menggunakan variabel \kesimpulan yang didefinisikan sebagai "Penutup" di config/settings.tex
\label{bab:6} % Melabeli ini sebagai Bab 6
%-----------------------------------------------------------------------------%
Dengan memanjatkan puji syukur kehadirat Tuhan Yang Maha Esa, penyusunan laporan penelitian \type\ yang berjudul "\judul" ini telah sampai pada bagian akhir. Seluruh rangkaian kegiatan penelitian, mulai dari identifikasi masalah, studi literatur, perancangan metodologi, implementasi sistem, hingga analisis hasil dan pembahasan, telah diuraikan secara komprehensif dalam bab-bab sebelumnya.

Pada Bab 5, telah dipaparkan secara rinci kesimpulan-kesimpulan utama yang ditarik dari keseluruhan hasil penelitian, keterkaitan temuan dengan tujuan dan pertanyaan penelitian yang telah dirumuskan, serta identifikasi keterbatasan-keterbatasan yang ada. Saran-saran untuk pengembangan dan penelitian selanjutnya juga telah disampaikan sebagai upaya untuk perbaikan dan eksplorasi lebih lanjut di masa mendatang.

Penulis berharap bahwa penelitian ini dapat memberikan kontribusi yang bermanfaat, baik secara teoretis bagi pengembangan ilmu pengetahuan di bidang kecerdasan buatan dan analisis data dalam konteks pendidikan, maupun secara praktis bagi institusi pendidikan dalam upaya menjaga dan meningkatkan integritas akademik di lingkungan pembelajaran daring. Semoga hasil penelitian ini dapat menjadi landasan bagi inovasi-inovasi selanjutnya dan memberikan inspirasi bagi peneliti lain yang memiliki minat serupa.

Akhir kata, penulis menyadari bahwa laporan penelitian ini masih jauh dari kesempurnaan. Oleh karena itu, kritik dan saran yang membangun senantiasa diharapkan demi penyempurnaan di masa yang akan datang. Semoga laporan ini dapat memenuhi syarat sebagai karya ilmiah dan memberikan manfaat bagi semua pihak yang membacanya.