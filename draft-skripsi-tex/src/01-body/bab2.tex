%-----------------------------------------------------------------------------%
\chapter{\babDua}
\label{bab:2}
%-----------------------------------------------------------------------------%

Bab ini menyajikan tinjauan pustaka yang komprehensif mengenai landasan teoretis dan penelitian terkait sistem deteksi kecurangan akademik berbasis kecerdasan buatan. Pembahasan mencakup evolusi masalah integritas akademik dalam pembelajaran daring, perkembangan teknik \textit{machine learning} untuk deteksi anomali, serta aplikasi spesifik dalam lingkungan \textit{Learning Management System} (LMS) seperti Moodle.

%-----------------------------------------------------------------------------%
\section{Integritas Akademik dalam Era Pembelajaran Daring}
\label{sec:integritasAkademik}
%-----------------------------------------------------------------------------%

\subsection{Evolusi Tantangan Integritas Akademik}
\label{subsec:evolusiTantangan}

Integritas akademik telah menjadi perhatian fundamental dalam dunia pendidikan sejak lama, namun transformasi digital pendidikan telah mengubah secara signifikan lanskap dan karakteristik permasalahan ini. Lanier \cite{Lanier2006} dalam penelitiannya yang menjadi rujukan penting, menemukan bahwa tingkat kecurangan dalam pembelajaran jarak jauh cenderung lebih tinggi dibandingkan dengan kelas tatap muka tradisional. Temuan ini menjadi dasar pemahaman bahwa lingkungan pembelajaran daring memerlukan pendekatan pemantauan yang berbeda dan lebih komprehensif.

Chirumamilla dkk. \cite{Chirumamilla2020} melalui survei terhadap mahasiswa dan dosen di Norwegia, mengidentifikasi enam modus kecurangan paling umum dalam ujian daring: (1) peniruan identitas (\textit{identity theft}), (2) penggunaan bahan bantuan terlarang, (3) kolaborasi tidak sah antarpeserta, (4) penggunaan perangkat komunikasi selama ujian, (5) akses ke sumber eksternal tanpa izin, dan (6) manipulasi waktu pengerjaan. Klasifikasi ini memberikan kerangka pemahaman yang sistematis tentang berbagai bentuk pelanggaran yang perlu dideteksi oleh sistem otomatis.

Penelitian terbaru menunjukkan bahwa pandemi COVID-19 telah mempercepat adopsi pembelajaran daring sekaligus meningkatkan kompleksitas permasalahan integritas akademik. Yulita dkk. \cite{Yulita2023} mengamati peningkatan signifikan dalam kecanggihan metode kecurangan, termasuk penggunaan teknologi untuk memfasilitasi kolusi dan berbagi informasi secara real-time selama ujian berlangsung.

\subsection{Karakteristik Kecurangan dalam Lingkungan Digital}

Lingkungan pembelajaran digital memiliki karakteristik unik yang membedakannya dari pengaturan tradisional. Murdoch dan House \cite{Murdoch2019} mengidentifikasi fenomena \textit{"ghost in the shell"}, yaitu perpaduan antara kecurangan berbasis kontrak (\textit{contract cheating}) dengan peniruan identitas daring. Fenomena ini menunjukkan evolusi kecurangan dari tindakan individual menjadi operasi yang lebih terorganisasi dan teknologis.

Balderas dan Caballero-Hern\'{a}ndez \cite{Balderas2020} dalam analisis mereka terhadap rekam jejak pembelajaran selama pandemi, menemukan bahwa pola perilaku mencurigakan dapat diidentifikasi melalui analisis temporal dan spasial aktivitas mahasiswa. Temuan ini memperkuat argumen bahwa data log aktivitas mengandung informasi yang kaya untuk deteksi kecurangan, asalkan dianalisis dengan metode yang tepat.

%-----------------------------------------------------------------------------%
\section{Pendekatan \textit{Machine Learning} untuk Deteksi Kecurangan}
\label{sec:mlApproaches}
%-----------------------------------------------------------------------------%

\subsection{Evolusi dari Sistem Berbasis Aturan ke \textit{Machine Learning}}

Pendekatan tradisional untuk deteksi kecurangan akademik umumnya mengandalkan sistem berbasis aturan (\textit{rule-based systems}) yang menggunakan ambang batas statis untuk mengidentifikasi perilaku mencurigakan. Huda dkk. \cite{article:rule_based_limitations} mengidentifikasi beberapa keterbatasan fundamental dari pendekatan ini: (1) rendahnya akurasi dalam menangani pola perilaku yang kompleks, (2) tingginya tingkat \textit{false positive} yang mengakibatkan banyak mahasiswa normal yang salah dituduh, (3) ketidakmampuan untuk beradaptasi dengan modus kecurangan yang berkembang, dan (4) kesulitan dalam menangani variasi kontekstual antarmata kuliah atau institusi.

Sebagai respons terhadap keterbatasan ini, penelitian modern beralih ke pendekatan berbasis \textit{machine learning} yang menawarkan kemampuan adaptif dan akurasi yang lebih tinggi. Kamalov dkk. \cite{Kamalov2021} menunjukkan bahwa model pembelajaran mesin dapat mencapai akurasi deteksi hingga 94\% dalam mengidentifikasi kecurangan ujian, jauh melampaui kinerja sistem berbasis aturan tradisional.

\subsection{Teknik \textit{Supervised Learning} untuk Deteksi Kecurangan}

Pendekatan pembelajaran terawasi (\textit{supervised learning}) telah menjadi metode dominan dalam deteksi kecurangan akademik karena kemampuannya dalam mempelajari pola dari data berlabel. Zhou dan Jiao \cite{Zhou2022} dalam penelitian mereka tentang augmentasi data untuk deteksi kecurangan dalam asesmen skala besar, mendemonstrasikan efektivitas berbagai algoritma pembelajaran terawasi, termasuk \textit{Random Forest}, \textit{Support Vector Machine}, dan \textit{Gradient Boosting}.

Alsabhan \cite{Alsabhan2023} mengembangkan pendekatan hibrida yang mengintegrasikan \textit{Long Short-Term Memory} (LSTM) dengan teknik pembelajaran mesin tradisional untuk mendeteksi kecurangan mahasiswa di perguruan tinggi. Penelitian ini menunjukkan bahwa kombinasi neural network dengan algoritma ensemble dapat meningkatkan akurasi deteksi hingga 96,8\%, dengan kemampuan khusus dalam menangkap pola temporal yang kompleks dalam perilaku mahasiswa.

Chang dan Chang \cite{Chang2023} melakukan studi komprehensif tentang deteksi kolusi dalam ujian, dengan fokus pada teknik pembelajaran mesin dan representasi fitur. Mereka menemukan bahwa \textit{feature engineering} yang tepat, terutama yang berkaitan dengan matriks kesamaan dan analisis graf, dapat secara signifikan meningkatkan kemampuan deteksi kolaborasi tidak sah antarpeserta ujian.

\subsection{Deteksi Anomali dan \textit{Unsupervised Learning}}

Meskipun pembelajaran terawasi menunjukkan kinerja yang baik, ketersediaan data berlabel yang berkualitas seringkali menjadi kendala dalam implementasi praktis. Sebagai alternatif, pendekatan deteksi anomali berbasis \textit{unsupervised learning} menawarkan solusi yang menjanjikan. Cen dkk. \cite{survey:anomaly_detection_edu} mengembangkan kerangka kerja untuk deteksi anomali tanpa pengawasan dalam sistem \textit{e-learning}, yang mampu mengidentifikasi pola perilaku yang tidak biasa tanpa memerlukan data berlabel sebelumnya.

Alexandron dkk. \cite{Alexandron2019} mengusulkan metode deteksi anomali tujuan umum untuk mengidentifikasi kecurangan dalam \textit{Massive Open Online Courses} (MOOC). Penelitian mereka menunjukkan bahwa teknik deteksi anomali seperti \textit{Isolation Forest} dan \textit{Local Outlier Factor} dapat efektif dalam mengidentifikasi pola perilaku yang menyimpang dari norma, dengan tingkat presisi yang dapat diterima untuk implementasi praktis.

%-----------------------------------------------------------------------------%
\section{Sistem Deteksi Khusus Platform Moodle}
\label{sec:moodleSpecific}
%-----------------------------------------------------------------------------%

\subsection{Karakteristik Data Log Moodle}

Moodle sebagai salah satu LMS yang paling banyak digunakan di dunia, menyediakan sistem pencatatan log yang komprehensif dan terstruktur. Mazza dan Dimitrova \cite{article:moodle_logs} dalam penelitian pionir mereka, menjelaskan bahwa log aktivitas Moodle mengandung informasi detail tentang setiap interaksi pengguna dengan platform, termasuk timestamp, tipe aktivitas, durasi, dan konteks akademik.

Data log Moodle memiliki beberapa karakteristik unik yang membuatnya sangat cocok untuk analisis \textit{machine learning}: (1) granularitas tinggi dalam perekaman aktivitas, (2) konsistensi format data lintas berbagai modul, (3) integrasi dengan konteks pembelajaran yang memungkinkan analisis berbasis mata kuliah, dan (4) kemampuan pelacakan yang mencakup tidak hanya aktivitas ujian tetapi juga pola belajar secara keseluruhan.

\subsection{Implementasi Sistem Deteksi Terintegrasi}

Shatnawi dkk. \cite{Shatnawi2024} mengembangkan sistem deteksi kecurangan ujian elektronik yang terintegrasi langsung dengan platform Moodle LMS. Sistem ini menggunakan pendekatan pembelajaran mesin dengan metode statistik yang mampu mencapai akurasi 100\% dalam mendeteksi berbagai jenis anomali, termasuk deteksi waktu respons yang tidak wajar, pola navigasi mencurigakan, dan aktivitas yang tidak konsisten dengan perilaku normal mahasiswa.

Moreno-Marcos dkk. \cite{MorenoMarcos2023} mengembangkan Statoodle, sebuah alat \textit{learning analytics} yang diintegrasikan langsung dengan Moodle untuk menganalisis aksi mahasiswa dan mencegah kecurangan. Sistem ini menggunakan pendekatan pemantauan waktu nyata yang dapat memberikan peringatan dini kepada pengawas ujian ketika terdeteksi aktivitas mencurigakan.

Pendekatan terintegrasi ini menawarkan beberapa keuntungan: (1) akses langsung ke data log tanpa perlu ekspor manual, (2) kemampuan pemantauan waktu nyata, (3) integrasi dengan alur kerja yang ada di institusi pendidikan, dan (4) kemudahan dalam implementasi tindakan preventif atau responsif.

\subsection{Analisis Kesamaan dan Deteksi Kolusi}

Salah satu kekuatan utama platform Moodle adalah kemampuannya dalam menyediakan data yang memungkinkan analisis kesamaan antarpeserta. Chang dan Chang \cite{Chang2023} menunjukkan bahwa analisis matriks kesamaan berbasis jawaban, pola navigasi, dan waktu dapat secara efektif mengidentifikasi kolusi antarmahasiswa.

Teknik analisis graf juga dapat diterapkan pada data Moodle untuk mengidentifikasi kluster mahasiswa yang menunjukkan pola perilaku yang serupa secara tidak wajar. Pendekatan ini tidak hanya dapat mendeteksi kecurangan individual, tetapi juga mengungkap jaringan kolaborasi yang lebih luas.

%-----------------------------------------------------------------------------%
\section{\textit{Learning Analytics} dan \textit{Educational Data Mining}}
\label{sec:learningAnalytics}
%-----------------------------------------------------------------------------%

\subsection{Evolusi \textit{Learning Analytics} sebagai Disiplin}

\textit{Learning analytics} telah berkembang menjadi disiplin yang matang dalam dekade terakhir, dengan fokus pada penggunaan data pendidikan untuk meningkatkan proses dan hasil pembelajaran. Siemens dan Long (2011) mendefinisikan \textit{learning analytics} sebagai pengukuran, pengumpulan, analisis, dan pelaporan data tentang pelajar dan konteks mereka, dengan tujuan memahami dan mengoptimalkan pembelajaran serta lingkungan tempat pembelajaran tersebut terjadi.

Dalam konteks deteksi kecurangan, \textit{learning analytics} menyediakan kerangka metodologis dan teknologis yang komprehensif. Ferguson (2012) mengidentifikasi bahwa pendekatan \textit{learning analytics} tidak hanya fokus pada deteksi masalah, tetapi juga pada pemahaman mendalam tentang pola perilaku belajar yang dapat membantu dalam pencegahan proaktif.

\subsection{Aplikasi \textit{Educational Data Mining} untuk Deteksi Anomali}

\textit{Educational Data Mining} (EDM) sebagai subbidang dari \textit{learning analytics} menyediakan teknik-teknik khusus untuk ekstraksi pola dari data pendidikan. Romero dan Ventura (2020) dalam ulasan komprehensif mereka, mengidentifikasi bahwa teknik EDM telah berkembang dari analisis deskriptif sederhana menjadi model prediktif yang kompleks.

Dalam konteks deteksi kecurangan, EDM menawarkan beberapa teknik yang relevan: (1) \textit{sequence mining} untuk menganalisis pola navigasi, (2) \textit{clustering} untuk mengidentifikasi grup mahasiswa dengan perilaku serupa, (3) \textit{association rule mining} untuk menemukan hubungan antaraktivitas, dan (4) \textit{classification} untuk membedakan perilaku normal dan mencurigakan.

\subsection{Integrasi Perspektif Pedagogis dan Teknologis}

Aspek penting dalam pengembangan sistem deteksi kecurangan adalah integrasi antara perspektif pedagogis dan teknologis. Ga\v{s}evi\'{c} dkk. (2015) menekankan bahwa sistem \textit{learning analytics} yang efektif harus mempertimbangkan tidak hanya aspek teknis deteksi, tetapi juga implikasi pedagogis dan etis dari implementasi sistem tersebut.

Dalam konteks deteksi kecurangan, hal ini berarti sistem harus dirancang tidak hanya untuk mengidentifikasi pelanggaran, tetapi juga untuk mendukung proses pembelajaran yang adil dan mendorong integritas akademik melalui pendekatan yang konstruktif daripada sekadar bersifat hukuman.

%-----------------------------------------------------------------------------%
\section{Teknik Ensemble dan Optimasi Model}
\label{sec:ensembleTechniques}
%-----------------------------------------------------------------------------%

\subsection{Pendekatan \textit{Ensemble Learning}}

\textit{Ensemble learning} telah terbukti sebagai salah satu pendekatan paling efektif dalam meningkatkan akurasi dan ketahanan model \textit{machine learning}. Dalam konteks deteksi kecurangan akademik, teknik ensemble menawarkan keuntungan khusus karena kemampuannya dalam menggabungkan kekuatan berbagai algoritma untuk menangani kompleksitas dan variasi pola kecurangan.

Zhou (2012) dalam "Ensemble Methods: Foundations and Algorithms" menjelaskan bahwa kekuatan ensemble terletak pada prinsip "diversity and accuracy", yaitu kombinasi model yang beragam namun akurat dapat menghasilkan kinerja yang superior dibandingkan model individual. Dalam konteks deteksi kecurangan, keberagaman ini sangat penting karena berbagai jenis kecurangan mungkin lebih baik dideteksi oleh algoritma yang berbeda.

\subsection{Strategi Integrasi Multialgoritma}

Penelitian terkini menunjukkan bahwa integrasi strategis antara model pembelajaran terawasi dengan teknik deteksi anomali dapat menghasilkan sistem yang lebih kuat. Nadeem dkk. \cite{Nadeem2024} dalam penelitian mereka tentang teknik pembelajaran mesin canggih untuk deteksi penipuan keuangan, menunjukkan bahwa kombinasi supervised dan unsupervised learning dapat mengatasi keterbatasan masing-masing pendekatan: supervised learning memberikan akurasi tinggi pada pola yang dikenal, sementara unsupervised learning dapat mendeteksi anomali baru yang belum pernah ditemui sebelumnya.

Dalam implementasi praktis, strategi ensemble dapat mencakup: (1) \textit{voting classifiers} yang menggabungkan prediksi beberapa model, (2) \textit{stacking} yang menggunakan meta-learner untuk mengoptimalkan kombinasi, (3) \textit{bagging} untuk mengurangi varians, dan (4) \textit{boosting} untuk mengurangi bias.

\subsection{Optimasi Ambang Batas dan Hiperparameter}

Optimasi ambang batas merupakan aspek kritis dalam sistem deteksi kecurangan karena pertukaran antara false positive dan false negative memiliki implikasi praktis yang signifikan. Ambang batas yang terlalu rendah akan menghasilkan banyak false positive yang dapat merugikan mahasiswa yang tidak bersalah, sementara ambang batas yang terlalu tinggi dapat membiarkan kecurangan lolos dari deteksi.

Niu dkk. \cite{Niu2025} dalam ulasan sistematis tentang metodologi pembelajaran mesin yang efektif menunjukkan bahwa optimasi ambang batas sebaiknya dilakukan dengan mempertimbangkan cost-sensitive learning, yaitu \textit{cost} dari berbagai jenis kesalahan diperhitungkan secara eksplisit. Dalam konteks akademik, \textit{cost} dari false positive (menuduh mahasiswa yang tidak bersalah) mungkin berbeda dengan \textit{cost} dari false negative (membiarkan pelanggar akademik lolos).

%-----------------------------------------------------------------------------%
\section{Analisis Matriks Kesamaan dan \textit{Graph-Based Detection}}
\label{sec:similarityAnalysis}
%-----------------------------------------------------------------------------%

\subsection{Teori Matriks Kesamaan dalam Deteksi Kolusi}

Analisis matriks kesamaan telah menjadi teknik fundamental dalam deteksi kolaborasi tidak sah dalam konteks akademik. Konsep ini didasarkan pada premis bahwa mahasiswa yang melakukan kolusi akan menunjukkan pola perilaku yang tidak natural serupa, baik dalam hal jawaban, pola navigasi, maupun waktu.

Ukuran kesamaan yang umum digunakan dalam konteks ini meliputi: (1) \textit{Cosine similarity} untuk mengukur kemiripan vektor fitur, (2) \textit{Jaccard similarity} untuk data kategorikal, (3) \textit{Pearson correlation} untuk hubungan linear, dan (4) \textit{Edit distance} untuk data sekuensial. Pemilihan ukuran yang tepat sangat bergantung pada jenis data dan karakteristik kecurangan yang ingin dideteksi.

\subsection{Analisis Graf dan \textit{Network Detection}}

Pendekatan berbasis graf menawarkan perspektif yang kuat untuk memahami pola kolaborasi dalam skala yang lebih besar. Dalam representasi graf, mahasiswa dapat dimodelkan sebagai simpul, sementara sisi merepresentasikan tingkat kesamaan atau kolaborasi yang dicurigai.

Teknik analisis graf yang relevan meliputi: (1) \textit{community detection} untuk mengidentifikasi kluster mahasiswa yang berkolaborasi, (2) \textit{centrality measures} untuk mengidentifikasi individu yang menjadi pusat dalam jaringan kolaborasi, (3) \textit{clustering coefficient} untuk mengukur tingkat interkoneksi, dan (4) \textit{modularity analysis} untuk memvalidasi struktur komunitas.

\subsection{Analisis Temporal dan \textit{Dynamic Networks}}

Aspek temporal dalam analisis kesamaan memberikan dimensi tambahan yang penting. Kecurangan seringkali menunjukkan pola temporal yang karakteristik, seperti pengiriman jawaban secara bersamaan, pola navigasi yang sinkron, atau perubahan jawaban yang terkoordinasi.

Analisis jaringan dinamis dapat mengungkap pola kolaborasi yang berkembang selama ujian berlangsung, memberikan wawasan yang tidak dapat diperoleh dari analisis statis. Hal ini sangat relevan untuk ujian yang berlangsung dalam periode waktu yang diperpanjang atau untuk menganalisis pola kecurangan lintas beberapa sesi.

%-----------------------------------------------------------------------------%
\section{Evaluasi dan Validasi Sistem Deteksi}
\label{sec:evaluationValidation}
%-----------------------------------------------------------------------------%

\subsection{Metrik Evaluasi dalam Konteks Akademik}

Evaluasi sistem deteksi kecurangan memerlukan pertimbangan khusus karena karakteristik unik dari domain akademik. Metrik evaluasi standar seperti akurasi, presisi, recall, dan F1-score tetap relevan, namun interpretasi dan prioritas mereka harus disesuaikan dengan konteks.

Dalam pengaturan akademik, false positive (menuduh mahasiswa yang tidak bersalah) memiliki implikasi yang sangat serius, termasuk kerusakan reputasi, stres psikologis, dan konsekuensi hukum yang potensial. Oleh karena itu, presisi menjadi metrik yang sangat kritis. Di sisi lain, false negative (membiarkan pelanggar akademik lolos) dapat merusak keadilan sistem evaluasi dan mengancam integritas akademik secara keseluruhan.

\subsection{Validasi Lintas Domain dan Generalisasi}

Salah satu tantangan utama dalam pengembangan sistem deteksi kecurangan adalah memastikan generalisasi lintas mata kuliah, institusi, dan konteks yang berbeda. Model yang dilatih pada satu set data mungkin tidak berkinerja baik pada konteks yang berbeda karena variasi dalam perilaku mahasiswa, struktur mata kuliah, atau format ujian.

Strategi validasi silang yang kuat perlu mempertimbangkan tidak hanya pembagian acak, tetapi juga stratifikasi berdasarkan jenis mata kuliah, tingkat mahasiswa, atau faktor temporal. Hal ini penting untuk memastikan bahwa model dapat menggeneralisasi ke situasi dunia nyata yang beragam.

\subsection{Aspek Etis dan Keadilan}

Implementasi sistem deteksi kecurangan otomatis menimbulkan berbagai pertanyaan etis yang perlu dipertimbangkan secara serius. Keadilan algoritma menjadi isu yang kritis, terutama terkait dengan bias potensial terhadap kelompok mahasiswa tertentu.

Aspek etis yang perlu dipertimbangkan meliputi: (1) transparansi dalam proses pengambilan keputusan, (2) dapat dijelaskannya prediksi yang dihasilkan, (3) keadilan lintas kelompok demografis yang berbeda, (4) perlindungan privasi dalam penanganan data, dan (5) pengawasan manusia dalam proses pengambilan keputusan final.

%-----------------------------------------------------------------------------%
\section{Kesenjangan Penelitian dan Peluang Pengembangan}
\label{sec:researchGaps}
%-----------------------------------------------------------------------------%

\subsection{Identifikasi Kesenjangan dalam Literatur}

Meskipun telah terdapat banyak penelitian dalam bidang deteksi kecurangan akademik, beberapa kesenjangan masih dapat diidentifikasi: (1) kurangnya penelitian yang fokus pada integrasi komprehensif antara beberapa teknik, (2) penelitian terbatas pada optimasi ambang batas untuk meminimalkan false positive dalam konteks asesmen akademik berisiko tinggi, (3) perhatian yang tidak memadai pada dinamika temporal dan evolusi pola kecurangan, dan (4) kurangnya standar pembanding untuk evaluasi komparatif.

\subsection{Peluang untuk Kontribusi Novel}

Penelitian ini memiliki peluang untuk memberikan kontribusi dalam beberapa area: (1) pengembangan kerangka kerja ensemble yang mengintegrasikan supervised learning, deteksi anomali, dan analisis kesamaan secara optimal, (2) inovasi dalam feature engineering berbasis matriks kesamaan yang dapat menangkap pola kolaborasi yang kompleks, (3) pengembangan optimasi ambang batas yang sadar konteks yang dapat beradaptasi dengan pengaturan akademik yang berbeda, dan (4) penciptaan kerangka kerja evaluasi komprehensif yang mempertimbangkan kinerja teknis dan implikasi praktis.

%-----------------------------------------------------------------------------%
\section{Ringkasan}
\label{sec:ringkasanBab2}
%-----------------------------------------------------------------------------%

Tinjauan pustaka ini menunjukkan bahwa deteksi kecurangan akademik dalam pembelajaran daring telah berkembang dari sistem berbasis aturan sederhana menjadi implementasi \textit{machine learning} yang canggih. Integrasi berbagai teknik analitik, termasuk \textit{supervised learning}, deteksi anomali, analisis matriks kesamaan, dan pendekatan ensemble, menawarkan potensi untuk mengembangkan sistem deteksi yang lebih akurat dan kuat.

Penelitian-penelitian yang diulas menunjukkan bahwa tidak ada teknik tunggal yang optimal untuk semua jenis perilaku kecurangan. Sebaliknya, pendekatan terintegrasi yang menggabungkan kekuatan algoritma yang berbeda sambil memitigasi kelemahan masing-masing menunjukkan hasil yang paling menjanjikan.

Platform Moodle, dengan sistem pencatatan log yang komprehensif dan kemampuan integrasi yang baik, menyediakan lingkungan yang ideal untuk implementasi sistem deteksi canggih. Data log yang kaya dan terstruktur memungkinkan ekstraksi fitur beragam yang dapat menangkap berbagai aspek perilaku mahasiswa.

Namun, implementasi praktis sistem deteksi otomatis juga menimbulkan tantangan terkait keadilan, etika, dan penerapan praktis. Oleh karena itu, pengembangan sistem yang tidak hanya andal secara teknis tetapi juga bertanggung jawab secara etis dan dapat diterapkan secara praktis menjadi fokus penting untuk penelitian selanjutnya.

Penelitian ini bertujuan untuk mengisi beberapa kesenjangan yang diidentifikasi dengan mengembangkan kerangka kerja komprehensif yang mengintegrasikan beberapa teknik deteksi sambil mempertimbangkan kendala praktis dan pertimbangan etis dalam konteks akademik.
